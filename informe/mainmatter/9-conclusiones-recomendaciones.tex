% called by main.tex
%
\chapter{Conclusiones \& Recomendaciones}
\label{ch::capitulo9}

Haciendo un balance del trabajo, podemos afirmar que logramos cumplir con los objetivos que nos planteamos en un comienzo, los cuales se encuentran detallados en la sección \hyperref[ch::capitulo3]{\textit{Objetivos del proyecto}}. 

En primer lugar, logramos comprender en detalle tanto el algoritmo de Random Forest como su implementación concreta en la librería \textit{Scikit-learn}, así como los conceptos y principios fundamentales que lo sustentan. Por otro lado, obtuvimos datos de entrenamiento con distribuciones similares a las observadas en los resultados de \cite{navajasAggregatedKnowledge} y le realizamos la ingeniería de datos necesaria para el correcto funcionamiento de nuestras propuestas y el algoritmo original de Random Forest. 

A partir de allí, ideamos propuestas de simulación de debate para las ideas que habíamos planteado en un inicio, como la exclusión de extremos en grupos con los modelos \texttt{IQRRandomForestRegressor} y \texttt{PercentileTrimmingRandomForestRegressor}, el uso del promedio como en el algoritmo original, pero editado para ser ponderado por una confianza, con el modelo \texttt{OOBRandomForestRegressor}, y la combinación de árboles, mediante la idea de unir los primeros cortes, implementada en \texttt{FirstSplitCombinerRandomForestRegressor}. 

A su vez,  a lo largo del proyecto y mientras implementamos estas propuestas, fueron surgiendo nuevas, como la combinación entre el promedio ponderado por confianza y la exclusión de extremos, lo cual implementamos en nuestro modelo \texttt{OOBPlusIQRRandomForestRegressor} y una nueva propuesta de conocimiento compartido, utilizada en el último modelo explorado \texttt{SharedKnowledgeRandomForestRegressor}.

Con todos estos modelos implementados, optimizamos cada uno de ellos realizando una gran barrida de hiperparámetros en rangos razonables, y finalmente, con todos los modelos implementados y optimizados, realizamos tests estadísticos y visualizaciones para compararlos entre sí y con el modelo original.

Si bien tras estos tests, encontramos que nuestras alternativas que simulan la etapa intermedia de “debate” no resultaron en modelos significativamente superiores al Random Forest original, sí pudimos encontrar que algunas de ellas obtienen resultados similares en promedio, siendo en algunos casos levemente mejores y en otros casos levemente inferiores. Esto, nos motiva a plantear posibles iteraciones para el futuro, como podría ser realizar una contribución a la librería open-source \textit{Scikit-learn} para que estas modificaciones queden como modelos alternativos a considerar para diferentes aplicaciones o instancias en las cuáles supere al state-of-art. Por su parte, al igual que discutido anteriormente, se podría probar qué sucede si se continúa con la optimización de las alternativas, utilizando los diversos hiperparámetros que posee el algoritmo original y que por una cuestión de tiempo, quedaron fuera del alcance de nuestro proyecto para evaluar. Adicionalmente, así como surgió el modelo de \texttt{OOBPlusIQRRandomForestRegressor} a partir de sumar ideas de simulación exploradas en otros modelos, de igual manera se podría seguir explorando otras combinaciones. Por ejemplo, se le podría sumar al modelo de conocimiento compartido, etapas posteriores a la construcción de árboles extendidos como la exclusión de extremos o la ponderación por confianza. 

También notamos que, si bien nuestras propuestas para el debate incluyen muchas formas distintas de combinar el conocimiento de los árboles, no tienen en cuenta el replanteo del problema desde cero que se puede dar durante la deliberación. Este aspecto es algo que podría ser fundamental para la mejora en las predicciones, dado que los seres humanos al debatir tienen la capacidad de barajar y dar de nuevo con la pregunta en cuestión e ir consensuando la respuesta, sin necesidad de combinar las respuestas originales que cada uno había pensado. Simulaciones nuevas de debate entre árboles que utilicen estos conceptos podrían ser interesantes de explorar y abren la puerta a nuevas líneas de investigación a la vez que surgen nuevos estudios del comportamiento humano que comprendan mejor el proceso de deliberación entre personas.

Por otro lado, podría ser también interesante explorar la simulación de otras consideraciones en el debate entre personas, en particular, el rol de la \textit{presión social} que provocan las opiniones de los demás a la hora de tomar una decisión. En nuestro caso, este aspecto podría ser tenido en cuenta en nuestro último modelo de conocimiento compartido simulando el impacto que tiene las predicciones de los otros árboles, análogamente a cómo afectan las opiniones de otras personas. Quizás, con las modificaciones necesarias, se podría, partiendo de la idea del modelo \texttt{SharedKnowledgeRandomForestRegressor}, simular el efecto de \textit{presión social}.

Más allá de las definiciones formales de la \textit{presión social}, se la puede pensar como un mayor peso de la información de los demás por sobre otra. Esto se podría efectivamente simular dándole un mayor peso a las features otorgadas por otros árboles del grupo, por ejemplo con una mayor probabilidad de ser seleccionadas para el corte, o dándoles una ventaja a la hora de compararse frente a otras alternativas de corte en cada nivel durante entrenamiento. Sería probable hacer que esta presión pueda ser ajustada mediante un hipotético nuevo hiperparámetro llamado “pressure”.

Finalmente, Random Forest es solo uno de los varios modelos de ensamble que existen. Nuevos enfoques de investigación podrían ser evaluar simulaciones con propuestas similares a las de este proyecto en otros modelos de ensamble, o si se encuentra algún mecanismo de agregación más sofisticado y eficiente que modele el debate.
