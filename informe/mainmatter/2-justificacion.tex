% called by main.tex
%
\chapter{Justificación del tema}
\label{ch::capitulo2}

Como vimos, muchos de los algoritmos y estructuras que se utilizan hoy en día en IA y machine learning son producto de intentar imitar el comportamiento del ser humano y su cerebro. Un claro ejemplo son los transformers de Google, que si bien no es necesario entender su funcionamiento, se basan en el concepto de “prestar atención al contexto”, algo que los seres humanos hacemos muy bien. Y no es necesario irnos a estructuras tan recientes. Los conceptos fundacionales de la IA como la conocemos hoy, nacieron del perceptrón o las redes neuronales de los cuales escuchamos hablar constantemente y son producto de buscar también imitar cómo se comporta nuestro cerebro.

En nuestro proyecto, se busca incluir las dinámicas observadas de inteligencia colectiva con personas en un algoritmo de machine learning llamado Random Forest. Este algoritmo, fue el puntapié que sentó las bases para muchos de los llamados \textbf{ensemble} models (en español, modelos de ensamble), ampliamente utilizados en la actualidad. Por esta razón, nos motiva pensar que, si encontrásemos que la inclusión de mecanismos de agregación de modelos mediante el “debate” da resultados positivos, sería una buena idea aplicar este mismo tipo de ideas a otros modelos state-of-art.

Si bien la idea de intentar replicar un efecto que ocurre en seres humanos en algoritmos de aprendizaje supervisado tiene valor a nivel académico, también lo puede tener a nivel práctico. En particular, si se necesitan realizar predicciones muy rápidamente, Random Forest podría ser una elección válida, por lo que una modificación al mismo que no reduzca su velocidad y aumente su performance sería muy valorable. Por otro lado, si la idea muestra potencial y en un futuro se aplica a modelos state-of-art dando mejoras, eso implica mejorar de cierta manera aquellas soluciones que se basan en dichos modelos.

La elección del tema también tiene un componente personal. A los tres integrantes de nuestro grupo nos parecen fascinantes los conceptos detrás de todo lo que es machine learning y sus posibles aplicaciones. Además, dada la carrera multidisciplinaria en la que nos embarcamos estos años, nos parece una increíble oportunidad aplicar los conocimientos computacionales junto con las temáticas de neurociencia y comportamiento humano abordadas también a lo largo de la misma. Adicionalmente, valoramos mucho la posibilidad de acercarnos, aunque sea un poco, al mundo de la investigación académica en una temática en la que consideramos estamos muy bien capacitados para abordar. 

En resumen, nos resulta un tema perfecto para nuestro equipo y consideramos es no sólo atractivo sino también relevante, dados los motivos ya previamente mencionados.
