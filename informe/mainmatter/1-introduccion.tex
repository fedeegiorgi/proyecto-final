% called by main.tex
%
\chapter{Introducción}
\label{ch::capitulo1}

En las últimas décadas, el área de la inteligencia artificial (IA) avanzó a un ritmo acelerado, revolucionando diversas disciplinas. Dentro de estos progresos, los algoritmos de aprendizaje automático se establecieron como herramientas esenciales para analizar datos, realizar predicciones y automatizar procesos complejos. Estos algoritmos fueron aplicados en diversos campos como la medicina, el ámbito financiero e incluso el psicológico.

Desde las primeras reflexiones de Alan Turing acerca de la posible creación de máquinas inteligentes, la evolución de la inteligencia artificial se caracterizó por la búsqueda de algoritmos que, de algún modo, simulen los procesos de pensamiento humano. Estas ideas de Turing abrieron la puerta al desarrollo de sistemas capaces de manejar datos de acuerdo a normas preestablecidas. Con el paso del tiempo, estas ideas evolucionaron en la creación de modelos de aprendizaje automático, que pueden aprender de manera automática de los datos sin requerir instrucciones exactas. En este escenario, los modelos de ensamble forman parte de un avance esencial, dado que utilizan varios modelos, o “opiniones”, para producir decisiones más precisas, de forma parecida a cómo los humanos pueden tomar decisiones más acertadas al tener en cuenta diferentes puntos de vista.

Uno de estos modelos de ensamble más reconocidos es Random Forest (RF), un algoritmo que combina varios árboles de decisión para generar predicciones. Esta perspectiva concuerda con ideas de las ciencias del comportamiento, como la “sabiduría de las masas”, en la que la suma de varias estimaciones individuales suele superar la precisión de cualquier estimación particular. 

El desarrollo de este tipo de algoritmos como RF fueron de la mano del auge de las neurociencias y las ciencias cognitivas, disciplinas que han buscado generar un conocimiento más profundo del comportamiento humano. La IA se nutrió en gran medida de estas ciencias, inspirándose para diseñar modelos de machine learning que imitan procesos humanos. 

Este trabajo se ubica justamente en la intersección entre la IA y las ciencias del comportamiento, para explorar, al igual que se realizó a lo largo de la historia, si se pueden simular observaciones del comportamiento humano en algoritmos como el RF para mejorar su rendimiento. Particularmente, inspirado en \cite{navajasAggregatedKnowledge}, el proyecto propone explorar estrategias de agregación que simulen el “debate” entre modelos de aprendizaje supervisado, específicamente árboles de decisión.

En este sentido, el proyecto se centra en la experimentación con el algoritmo de Random Forest, buscando responder preguntas como: ¿cómo se pueden modelar los procesos de debate entre humanos vistos en Navajas en este algoritmo? ¿Estas nuevas formas de agrupamiento mejorarán la precisión de las predicciones? Responderlas es el objetivo del trabajo a través de implementaciones de variantes del algoritmo en cuestión.
